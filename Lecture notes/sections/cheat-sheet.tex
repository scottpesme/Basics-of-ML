\section{Useful Formulas and Good Practices}

\subsection{Definitions and notations}

Keep in mind that these are the notations I like to use, but these are obviously personal and others will use different ones!!
\begin{itemize}
    \item the notation $x$ will always be used for input data. E.g. $x_1, \dots, x_n \in \R^d$ could be some dataset. $n$ is then the number of training samples, and $d$ the dimension of each data point. Similarly, $y$ will used for output data, e.g. $y_1, \dots, y_n \in \R$ (or $\in \{0, 1\}$) are output data (or labels). When $d > n$, we often say that we are in the \textit{over-parametrised setting} (also called \textit{under-determined setting}, which can be confusing!), if $d < n$ we say that we are in the  \textit{under-parametrised setting} (also called, \textit{under-determined setting}).
    \item $X = \begin{pmatrix}
- & x_1 & - \\
  & \vdots &  \\
- & x_n & -
\end{pmatrix}
\in \mathbb{R}^{n \times d}$ corresponds to data / feature / design /observation matrix. It has $n = \textit{"number \ of \ samples"}$ rows and $d = \textit{"dimension \ of \ datapoints"}$ columns.
    \item $y = (y_1, \dots, y_n) \in \R^{n}$ corresponds to the output vector.
    \item for a vector $u \in \R^d$, we let $\Vert u \Vert_2 \coloneqq \sqrt{\sum_{i = 1}^d u_i^2}$ denote the Euclidean norm of $u$ (also called $\ell_2$-norm).
    \item for vectors $u, v \in \R^d$, we denote $\langle u, v \rangle \coloneqq \sum_{i = 1}^d u_i, v_i$ to be the inner product of $u$ and $v$. Notice that $\Vert u \Vert^2 = \langle u, u \rangle$. Two vectors are said to be orthogonal if their inner product is null.
    \item for vectors $x_1, \dots, x_n$, their span corresponds to the linear space which they generate: $\span(x_1, \dots, x_n) \coloneqq \{ \lambda_1 x_1 + \dots, \lambda_n x_n , \text{where} \  \lambda_1, \dots, \lambda_n \in \R \}$.
    \item the rank of a matrix corresponds to the dimension of the span of its columns, which is equal to the dimension of the span of its rows. Therefore for $X \in \R^{n \times d}$, it holds that $\rank(X) \leq \min(n, d)$. A matrix is said to be full rank if $\rank(X) = \min(n, d)$.
    \item the null space (sometimes called kernel) of a matrix $A \in \R^{n \times d}$ is defined as $\ker(A) = \{ w \in \R^d, A w = 0 \}$.
    \item a matrix $A \in \R^{d \times d}$ is said to be symmetric if $A^{\top} = A$. It is said to be positive semi-definite (we write this as  $A \succeq 0$) if for all vector $u \in \R^d$, $u^\top A u \geq 0$. This is equivalent to saying that all the eigenvalues of $A$ are positive. 
\end{itemize}

\subsection{Gradients}

% \begin
If $f : \R^d \to \R$ is twice differentiable, then its gradient $\nabla f : \R^d \to \R^d$ and hessian $\nabla^2 f : \R^d \to \R^{d \times d}$ are defined as: 
\[\nabla f(w) = \Big ( \frac{\partial f}{\partial w_i}(w) \Big )_{1 \leq i \leq d} \quad \nabla^2 f(w) = \Big ( \frac{\partial^2 f}{\partial w_i \partial w_i}(w) \Big )_{1 \leq i, j \leq d} \] 

\begin{itemize}
    \item For a vector $b \in \R^d$, the gradient of the linear function $f : \R^d \to \R$, $f(w) = \langle a , w \rangle $ is equal to $\nabla f(w) = b$. 
    \item For a (not necessarily symmetric) matrix $A \in \R^{d \times d}$, $f(w) = \frac{1}{2} w^\top A w$ is a quadratic function. Its gradient is $\nabla f(w) = \frac{1}{2} (A + A^\top) w$, which is equal to $A w$ if and only if $A$ is a symmetric matrix. The hessian of $f$ is equal to $\nabla^2 f(w) = \frac{1}{2} (A + A^\top)$. 
\end{itemize}

\subsection{Linear Algebra}

\begin{itemize}
    \item For a matrix $A \in \R^{d \times d}$, it holds that $w^\top A w = \sum_{i, j = 1}^d w_i w_j A_{i, j}$. 
    \item It holds that $X^\top X = \sum_{i=1}^n x_i x_i^\top \in \R^{d \times d}$ and $X X^\top = \big (\langle x_i, x_j \rangle \big )_{1 \leq i, j \leq n} \in \R^{n \times n}$.
    \item Let $L(w) = \frac{1}{2} \sum_{i = 1}^n (y_i - \langle w, x_i \rangle)^2$, then $L(w) = \frac{1}{2} \Vert y - X w \Vert^2$ and $\nabla L(w) = X^\top (X w - y)$.
    \item if $n < d$ ("underparametrised setting), then the matrix $X^\top X$ cannot be invertible, because $\span (x_1, \dots, x_n)$ cannot be equal to $\R^d$. However, if $n \geq d$ and $\span (x_1, \dots, x_n) = \R^d$, then $X^\top X$ is invertible.
    \item \textit{Rank–nullity theorem:} for $A \in \R^{n \times d}$, it holds that $\rank(A) + \dim \ker(A) = d$.
\end{itemize}

\subsection{Convexity}

A function $f:\R^d \to \R$ is said to be convex if for all $w_1, w_2 \in \R^d$ and $\lambda \in [0, 1]$, $f(\lambda w_1 + (1 - \lambda) w_2) \leq \lambda f(w_1) + (1 - \lambda) f(w_2)$ (DO A DRAWING TO VISUALISE THIS!). 

\begin{itemize}
    \item if $f$ is convex and differentiable, then for all $w_1, w_2 \in \R^d$, it holds that $f(w_2) \geq f(w_1) + \langle \nabla f(w_1), w_2 - w_1 \rangle$ (DO A DRAWING TO VISUALISE THIS!).
    \item if $f$ is convex, then all local minima are global. Therefore, if $\nabla f(w^\star) = 0$, then $w^\star$ is a global minimum. However, keep in mind that there exist convex functions which do not have any minima! (the exponential function for example)
    \item if a function $f$ is convex, then its sublevel sets $\{ w \in \R^d, f(w) \leq c \}$ are convex sets for all $c \in \R$. The converse is false (for example think of $x \mapsto \sqrt{|x|}$ in 1d).
    \item let $\lambda_1, \lambda_2 \geq 0$, if $f_1$ and $f_2$ are convex functions, then so is $\lambda_1 f_1 + \lambda_2 f_2$.
    \item $f : \R^n \to \R$ is convex and $A \in \R^{n \times d}$ and $b \in \R^n$, then $w \in \R^d \mapsto f(A w + b)$ is convex too. 
\end{itemize}

\subsection{Good practices and sanity checks}

\paragraph{Math sanity checks} Always check that the math makes sense! 

\begin{itemize}
    \item  If $f : \R^d \to \R$, then for a vector $w \in \R^d$, $\nabla f(w)$ must belong to $\R^d$! So for example if $f(w) = \frac{1}{2} \Vert w \Vert^2$, then writing that "$\nabla f(w) = \Vert w \Vert$" doesn't make sense. 
    \item Check that the matrix operations are allowed: if $L(w) = \Vert y - X w \Vert^2$, writing that "$\nabla L(w) = X (X w -y)$" doesn't make sense because the operations $X X$ and $X y$ don't make sense for $n \neq d$ (and also because of the remark right above).
\end{itemize}


\paragraph{Dimensional sanity check.}  
Even if an expression is mathematically well-formed, it might be meaningless from the point of view of ``units'' or ``dimensions.''  
A quick check is to make sure your formulas are \emph{homogeneous}: every term you add or compare should have the same ``type.'' 
 
\noindent
\textit{Example.} Suppose our data are temperature observations $y_i$ measured in degrees Celsius. A feature vector $x \in \mathbb{R}^d$ could represent input quantities such as:
\[
x = \big([\text{altitude in meters}], [\text{pressure in pascals}], [\text{wind speed in meters per second}] \big).
\]
A parameter vector $w \in \mathbb{R}^d$ scales each feature so that the inner product $\langle w, x \rangle$
has the same unit as $y$ (degrees Celsius). In our small example the units of $w$ are 
\[ ([^\circ\mathrm{C}] \cdot [\text{meters}]^{-1}, [^\circ\mathrm{C}] \cdot [\text{pascals}]^{-1}, [^\circ\mathrm{C}] \cdot [\text{meters}]^{-1} \cdot [\text{seconds}]) \]
Now notice that:
\begin{itemize}
    \item $\langle w, x \rangle$ makes sense: each coordinate of $w$ carries the reciprocal unit of the corresponding coordinate of $x$, so the sum yields something in degrees Celsius.  
    \item $w + x$ does \emph{not} make sense: $w$ and $x$ do not have the same units (adding ``degrees per meter'' to ``meters'' is meaningless).  
\end{itemize}


